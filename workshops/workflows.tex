%! suppress = FileNotFound
%! suppress = MissingImport
\RequirePackage{import}
\subimport{../common}{preamble}
\subimport{../common}{packages}
\subimport{../common}{vars}
\begin{document}
    \begin{frame}[c]
        \centering
        \Large
        \textbf{Effektive Git Workflows, Teamarbeit \& Best Practices}
    \end{frame}


    \section{Intro}\label{sec:intro}
    \begin{frame}[c]
        \slidehead
        \vspace{-1em}
        \centering
        \large
        \textbf{Das Heutige Programm}
        \vspace{1em}
        \begin{itemize}[<+->]
            \item Quick Recap
            \item Git Workflows \& Practices
            \begin{itemize}
                \item Centralized Workflow
                \item Feature Branch Workflow
                \item Kurzer Einblick in andere Workflows
            \end{itemize}
            \item CI/CD
            \item Git Platforms
        \end{itemize}
    \end{frame}

    \subsection{Git Recap}\label{subsec:git-recap}
    \begin{frame}[c]
        \slidehead
        \vspace{-1em}
        \centering
        \large
        \textbf{Das Git-VCS}
        \vspace{1em}
        \begin{itemize}[<+->]
            \item Dezentrales Version Control System
            \item Ein \textbf{Commit} ist ein Zeiger auf einen Zustand vom Repository
            \item Eine \textbf{Branch} ist ein Zeiger auf einen \textbf{Commit}
        \end{itemize}
    \end{frame}

    \begin{frame}[c]
        \slidehead
        \vspace{-1em}
        \centering
        \large
        \textbf{Git Operations}
        \vspace{2em}
        \begin{description}[<+->][labelwidth=\widthof{\bfseries The longest label}]
            \item [git merge] Kombiniert zwei Branches
            \item [git squash merge] Merged nur den letzten Zustand der Branch
            \item [git rebase] Fügt neue commits hinter den commits der aktuellen Branch
            \item [git cherry-pick] Kopiert einen commit von einer anderen Branch
            \item [git revert] Erstellt einen Commit der einen anderen Commit rückgängig macht
        \end{description}
    \end{frame}

%    \subimport{../chapters/workflows}{intro}
%    \subimport{../chapters/workflows}{centralized}
    \subimport{../chapters/workflows}{feature-branch}


    \section{Wie arbeitet man im Team?}\label{sec:team}


    \section{Best Practices}\label{sec:best-practices}


    \section{Git Platforms}\label{sec:git-platforms}

    \subimport{../chapters}{ci-cd}
\end{document}
