\section{Feature Branch Workflow}\label{sec:feature-branch-workflow}
\begin{frame}[c]
    \centering
    \Large
    \textbf{Feature Branch Workflow}
\end{frame}

\subsection{Der Workflow}\label{subsec:der-workflow}
\begin{frame}[c]
    \slidehead
    \centering
    \large
    \textbf{Feature Branch Workflow}
    \vspace{1em}
    \begin{enumerate}[<+->]
        \item \textbf{Feature Branch von Master abbranchen}
        \item \textbf{Feature implementieren}
        \item \textbf{Feature Branch mergen}
        \item \textbf{Feature Branch löschen}
    \end{enumerate}
\end{frame}

\subsection{Teamarbeit}\label{subsec:teamarbeit}
\begin{frame}[c]
    \slidehead
    \centering
    \begin{tikzpicture}
        \node[tiny-commit] at (0, 0)(a){};
        \node[tiny-commit] at (2, 0) (b){};
        \node[tiny-commit] at (4, 0) (c){};
        \node<2-5>[tiny-commit, fill=TUDa-8a] at (6, -1) (feature/a/1){};
        \node<3-6>[tiny-commit, fill=TUDa-10a] at (6, -2) (feature/b/1){};
        \node<4->[tiny-commit] at (6, 0) (d){};
        \node<5->[tiny-commit, fill=TUDa-8a!50] at (8, 0) (e){};
        \node<7>[tiny-commit, fill=TUDa-10a] at (10, -2) (feature/b/1){};
        \node<8>[tiny-commit, fill=TUDa-10a!50] at (10, 0) (f){};
        \node[branch] at (13, 0) (master){master};
        \node<2-5>[branch] at (13, -1) (feature/a){feature/a};
        \node<3-7>[branch] at (13, -2) (feature/b){feature/b};
        \draw[parent] (b) to (a);
        \draw[parent] (c) to (b);
        \draw<2-5>[parent] (feature/a/1) to (c);
        \draw<3-6>[parent] (feature/b/1) to (c);
        \draw<7>[parent] (feature/b/1) to (e);
        \draw<4->[parent] (d) to (c);
        \draw<5->[parent] (e) to (d);
        \draw<8>[parent] (f) to (e);
        \draw<-3>[ref-arc] (master) to (c);
        \draw<4>[ref-arc] (master) to (d);
        \draw<5-7>[ref-arc] (master) to (e);
        \draw<8>[ref-arc] (master) to (f);
        \draw<2-5>[ref-arc] (feature/a) to (feature/a/1);
        \draw<3-7>[ref-arc] (feature/b) to (feature/b/1);
    \end{tikzpicture}

    \vspace*{\fill}

    \begin{onlyenv}<2>
        \bashcommandbob{git checkout -b feature/a \&\& git commit}
    \end{onlyenv}
    \begin{onlyenv}<3>
        \bashcommandalice{git checkout -b feature/b \&\& git commit}
    \end{onlyenv}
    \begin{onlyenv}<5>
        \bashcommandbob{git merge \textminus\textminus squash feature/a}
    \end{onlyenv}
    \begin{onlyenv}<6>
        \bashcommandbob{git branch -d feature/a}
    \end{onlyenv}
    \begin{onlyenv}<7>
        \bashcommandalice{git rebase master}
    \end{onlyenv}
    \begin{onlyenv}<8>
        \bashcommandalice{git merge \textminus\textminus squash feature/b \&\& git branch -d feature/b}
    \end{onlyenv}
\end{frame}

\subsection{Final Review}\label{subsec:final-review}
\begin{frame}[c]
    \slidehead
    \centering
    \Large
    \textbf{Final Review}
\end{frame}

\begin{frame}[c]
    \slidehead
    \large
    \textbf{Bewertungskriterien}
    \normalsize
    \begin{enumerate}
        \item<2-> Wie skaliert der Workflow? \only<3->{\hfill\textbf{gut}}
        \item<4-> Können Bugs/Fehler einfach in den Master-Branch gelangen? \only<5->{\hfill\textbf{nein}}
        \item<6-> Kann man Fehler einfach rückgängig machen? \only<7->{\hfill\textbf{ja}}
        \item<8-> Erzeugt dieser Workflow eine neue unnötige, kognitive Überlastung für das Team? \only<9->{\hfill\textbf{it depends}}
    \end{enumerate}
\end{frame}
