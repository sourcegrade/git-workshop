\section{Git Platforms}\label{sec:git-platforms}
\begin{frame}[c]
    \centering
    \Large
    \textbf{Git Platforms}
\end{frame}

\subsection{GitHub}\label{subsec:github}
\begin{frame}[c]
    \slidehead
    \centering
    \large
    \textbf{GitHub}
    \vspace{1em}
    \begin{itemize}[<+->]
        \item \textbf{Git-Hosting-Service}
        \item \textbf{Kostenlos für öffentliche und private Repositories}
        \item \textbf{Kostenpflichtige Enterprise-Version}
        \item \textbf{Kostenpflichtige GitHub Pro-Version (free for education)}
        \item \textbf{Self-Hosting ist kostenpflichtig}
        \item \textbf{Closed Source}
        \item \textbf{Von Microsoft gekauft}
    \end{itemize}
\end{frame}

\subsection{GitLab}\label{subsec:gitlab}
\begin{frame}[c]
    \slidehead
    \centering
    \large
    \textbf{GitLab}
    \vspace{1em}
    \begin{itemize}[<+->]
        \item \textbf{Git-Hosting-Service}
        \item \textbf{Deutlich kleiner als GitHub aber wächst}
        \item \textbf{Kostenlos für öffentliche und private Repositories}
        \item \textbf{Kostenpflichtige Enterprise-Version}
        \item \textbf{Self-Hosting ist gratis (eine bezahlte Version gibt es auch)}
        \item \textbf{Open Source}
        \item \textbf{Von GitLab Inc.}
    \end{itemize}
\end{frame}

\subsection{Andere Plattformen}\label{subsec:andere-plattformen}
\begin{frame}[c]
    \slidehead
    \centering
    \large
    \textbf{Andere Plattformen}
    \vspace{1em}
    \begin{itemize}[<+->]
        \item \textbf{Bitbucket}
        \item \textbf{Gitee}
    \end{itemize}
\end{frame}
