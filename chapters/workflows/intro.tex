%! suppress = MissingImport


\section{Git Workflows}\label{sec:git-workflows}
\begin{frame}[c]
    \centering
    \Large
    \textbf{Git Workflows}
    \vspace{2em}
    \linebreak
    \begin{tikzpicture}
        \node[commit] at (0, 0)(a){A1A1A1};
        \node[commit] at (4, 0) (b){B2B2B2};
        \node[commit] at (4, -2) (c){C3C3C3};
        \node[branch] at (8, 0) (master){master};
        \node[branch] at (8, -2) (feature){feature};
        \draw[parent] (b) to (a);
        \draw[parent] (c.west) to (a.east);
        \draw[ref-arc] (master) to (b);
        \draw[ref-arc] (feature) to (c);
        \draw[parent, double] (feature) to (master);
    \end{tikzpicture}
\end{frame}

\subsection{Was ist ein Git Workflow?}\label{subsec:was-ist-ein-git-workflow?}
\begin{frame}[c]
    \slidehead
    \vspace{-1em}
    \centering
    \large
    \textbf{Was ist ein Git Workflow?}
    \vspace{2em}
    \only<2->{
        \linebreak
        \textit{Ein Git Workflow ist eine Rezeptur oder Empfehlung zur Verwendung von Git,
            die eine konsistente und produktive Arbeitsweise ermöglichen soll}
        \vspace{2em}
        \linebreak
        \textbf{
            \href{https://www.atlassian.com/de/git/tutorials/comparing-workflows}{Atlassian}
        }
    }
\end{frame}

\begin{frame}[c]
    \centering
    \large
    \textbf{Wie sieht ein erfolgreicher Git Workflow aus?}
\end{frame}

\subsection{Wie sieht ein erfolgreicher Git Workflow aus?}\label{subsec:wie-sieht-ein-erfolgreicher-git-workflow-aus?}
\begin{frame}[c]
    \slidehead
    \large
    \textbf{Bewertungskriterien}
    \begin{enumerate}
        \item<2-> Wie skaliert der Workflow?
        \item<3-> Kann man Fehler einfach rückgängig machen?
        \item<4-> Erzeugt dieser Workflow eine neue unnötige, kognitive Überlastung für das Team?
    \end{enumerate}
\end{frame}

\begin{frame}[c]
    \begin{columns}[c]
        \begin{column}{.5\textwidth}
            \centering
            \includesvg{../../pictures/bob}
        \end{column}
        \begin{column}{.5\textwidth}
            \Large
            \textbf{Meet Bob}
            \only<2->{
                \begin{itemize}
                \item<2-> Bob ist ein Softwareentwickler
                \item<3-> Bob probiert verschiedene Git Workflows aus
                \item<4-> Bob möchte einen Workflow finden, der zu ihm passt
                \item<5-> Bob arbeitet alleine
                \end{itemize}
            }
        \end{column}
    \end{columns}
\end{frame}
