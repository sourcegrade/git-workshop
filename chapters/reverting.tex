\section{Reverting}\label{sec:reverting}

\begin{frame}[b]
    \begin{center}
        \fontsize{48pt}{48pt}
        \faUndo
    \end{center}
    \vfill
    \begin{flushleft}
        \Huge
        \textbf{Reverting}
    \end{flushleft}
\end{frame}

\begin{frame}
    \slidehead
    \centering
    Ein revert ist ein neuer Commit, der die Änderungen eines anderen Commits rückgängig macht.
    \\
    \vspace{1em}
    \begin{tikzpicture}
        % state 1
        \node[commit](c1){c1c1c};
        \node[commit, right=2 em of c1](c2){c2c2c};
        \node[commit, right=2 em of c2](c3){c3c3c};
        \draw[parent] (c2) to (c1);
        \draw[parent] (c3) to (c2);

        \node[head, below=2 em of c3](head){};
        \draw[parent] (head) to (c3);

        % revert HEAD
        \node<2>[commit, fill=TUDa-9d, right=2 em of c3,label={[anchor=south,font=\small]above:Revert \texttt{c3c3c}}] (c4){c4c4c};
        \draw<2>[parent] (c4) to (c3);

        \node<4->[head-base, below=2 em of c2](head-1){HEAD\~{}1};
        \draw<4->[parent] (head-1) to (c2);

        % revert HEAD~1
        \node<3-4>[commit, fill=TUDa-9d, right=2 em of c3,label={[anchor=south,font=\small]above:Revert \texttt{c2c2c}}] (c5){c5c5c};
        \draw<3-4>[parent] (c5) to (c3);

        \node<5>[head-base, below=2 em of c1](head-2){HEAD\~{}2};
        \draw<5>[parent] (head-2) to (c1);

        % revert HEAD~2
        \node<5>[commit, fill=TUDa-9d, right=2 em of c3,label={[anchor=south,font=\small]above:Revert \texttt{c1c1c}}] (c6){c6c6c};
        \draw<5>[parent] (c6) to (c3);
    \end{tikzpicture}
    \\
    \vspace{1em}\par
    \only<2>{\bashcommand{git revert HEAD}}
    \only<3>{\bashcommand{git revert c2c2c}}
    \only<4>{\bashcommand{git revert HEAD\~{}1}}
    \only<5>{\bashcommand{git revert HEAD\~{}2}}
\end{frame}
