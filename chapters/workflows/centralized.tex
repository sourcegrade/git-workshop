%! suppress = MissingImport


\section{Centralized Workflow}\label{sec:centralized-workflow}
\begin{frame}[c]
    \centering
    \Large
    \textbf{Centralized Workflow}
\end{frame}

\subsection{Der Workflow}\label{subsec:der-workflow}
\begin{frame}[c]
    \slidehead
    \centering
    \large
    \textbf{Centralized Workflow}
    \vspace{1em}
    \begin{enumerate}[<+->]
        \item \textbf{Commit} auf den \textbf{Master}
        \item \textbf{Push} auf den \textbf{Master}
    \end{enumerate}
\end{frame}

\subsection{Normalfall}\label{subsec:normalfall}
\begin{frame}[c, fragile]
    \slidehead
    \begin{center}

        \Large

        \vspace{-1.5em}

        \textbf{Remote}

        \vspace{.5em}
        \only<1>{\vspace{.5em}}

        \normalsize

        \begin{tikzpicture}
            \node[tiny-commit] at (0, 0)(a){};
            \node[tiny-commit] at (2, 0) (b){};
            \node[tiny-commit] at (4, 0) (c){};
            \node<4>[tiny-commit, fill=TUDa-8a] at (6, 0) (d){};
            \node<-3>[branch] at (8, 0) (master){origin/master};
            \node<4>[branch] at (10, 0) (master){origin/master};
            \draw[parent] (b) to (a);
            \draw[parent] (c) to (b);
            \draw<4>[parent] (d) to (c);
            \draw<-3>[ref-arc] (master) to (c);
            \draw<4>[ref-arc] (master) to (d);
        \end{tikzpicture}
        \begin{onlyenv}<2->
            \rule{\textwidth}{.3mm}
            \Large

            \textbf{Local}

            \vspace{.5em}

            \normalsize
            \begin{tikzpicture}
                \node[tiny-commit] at (0, 0)(a){};
                \node[tiny-commit] at (2, 0) (b){};
                \node[tiny-commit] at (4, 0) (c){};
                \node<3->[tiny-commit, fill=TUDa-8a] at (6, 0) (d){};
                \node<2>[branch] at (8, 0) (master){master};
                \node<3->[branch] at (10, 0) (master){master};
                \draw[parent] (b) to (a);
                \draw[parent] (c) to (b);
                \draw<3->[parent] (d) to (c);
                \draw<2>[ref-arc] (master) to (c);
                \draw<3->[ref-arc] (master) to (d);
            \end{tikzpicture}

            \vspace{1em}

            \begin{onlyenv}<2>
                \bashcommandbob{git pull}
            \end{onlyenv}
            \begin{onlyenv}<3>
                \bashcommandbob{git commit}
            \end{onlyenv}
            \begin{onlyenv}<4>
                \bashcommandbob{git push}
            \end{onlyenv}
            \vspace{-1em}
        \end{onlyenv}
    \end{center}
\end{frame}

\subsection{Normallfall Review}\label{subsec:normallfall-review}
\begin{frame}[c]
    \slidehead
    \centering
    \Large
    \textbf{Review}
\end{frame}

\begin{frame}[c]
    \slidehead
    \large
    \textbf{Bewertungskriterien}
    \normalsize
    \begin{enumerate}
        \item<2-> Wie skaliert der Workflow? \only<3->{\hfill\textbf{unklar}}
        \item<4-> Können Bugs/Fehler einfach in den Master-Branch gelangen? \only<5->{\hfill\textbf{ja}}
        \item<6-> Kann man Fehler einfach rückgängig machen? \only<7->{\hfill\textbf{geht}}
        \item<8-> Erzeugt dieser Workflow eine neue unnötige, kognitive Überlastung für das Team? \only<9->{\hfill\textbf{nein}}
    \end{enumerate}
\end{frame}

\begin{frame}[c]
    \begin{columns}[c]
        \begin{column}{.5\textwidth}
            \centering
            \includesvg[scale=5]{../../pictures/alice}
        \end{column}
        \begin{column}{.5\textwidth}
            \Large
            \textcolor{TUDa-10a}{\textbf{Meet Alice}}
            \only<2->{
                \begin{itemize}
                \item<2-> Alice ist eine Softwareentwicklerin
                \item<3-> Alice arbeitet zusammen mit Bob
                \end{itemize}
            }
        \end{column}
    \end{columns}
\end{frame}

\subsection{Teamarbeit}\label{subsec:teamarbeit}
\begin{frame}[c, fragile]
    \slidehead
    \begin{center}

        \Large

        \vspace{-1.5em}

        \textbf{Remote}

        \vspace{.5em}
        \only<1>{\vspace{.5em}}

        \normalsize

        \begin{tikzpicture}
            \node[tiny-commit] at (0, 0)(a){};
            \node[tiny-commit] at (2, 0) (b){};
            \node[tiny-commit] at (4, 0) (c){};
            \node<4->[tiny-commit, fill=TUDa-10a] at (6, 0) (d){};
            \node<-3>[branch] at (8, 0) (master){origin/master};
            \node<4->[branch] at (10, 0) (master){origin/master};
            \draw[parent] (b) to (a);
            \draw[parent] (c) to (b);
            \draw<4->[parent] (d) to (c);
            \draw<-3>[ref-arc] (master) to (c);
            \draw<4->[ref-arc] (master) to (d);
        \end{tikzpicture}
        \rule{\textwidth}{.3mm}
        \Large

        \textbf{Local}

        \vspace{.5em}

        \normalsize
        \begin{tikzpicture}
            \node[tiny-commit] at (0, 0)(a){};
            \node[tiny-commit] at (2, 0) (b){};
            \node[tiny-commit] at (4, 0) (c){};
            \node<2,5>[tiny-commit, fill=TUDa-8a] at (6, 0) (d){};
            \node<3-4>[tiny-commit, fill=TUDa-10a] at (6, 0) (d){};
            \node<1>[branch] at (8, 0) (master){master};
            \node<2->[branch] at (10, 0) (master){master};
            \draw[parent] (b) to (a);
            \draw[parent] (c) to (b);
            \draw<2->[parent] (d) to (c);
            \draw<1>[ref-arc] (master) to (c);
            \draw<2->[ref-arc] (master) to (d);
        \end{tikzpicture}

        \vspace{1em}

        \begin{onlyenv}<1>
            \bashcommandbob{git pull}
        \end{onlyenv}
        \begin{onlyenv}<2>
            \bashcommandbob{git commit}
        \end{onlyenv}
        \begin{onlyenv}<3>
            \bashcommandalice{git commit}
        \end{onlyenv}
        \begin{onlyenv}<4>
            \bashcommandalice{git push}
        \end{onlyenv}
        \begin{onlyenv}<5>
            \bashcommandbob{git push \textcolor{TUDa-9a}{ERROR!}}
        \end{onlyenv}
        \vspace{-1em}
    \end{center}
\end{frame}

\begin{frame}[c]
    \slidehead
    \centering
    \large
    \textbf{Was ist passiert?}
    \small
    \begin{onlyenv}<2>
        \inputCode[]{
            title=\codeBlockTitle{Rejected Push},
        }{\codeDir/rejected-push.txt}
    \end{onlyenv}
\end{frame}

\begin{frame}[c, fragile]
    \slidehead
    \begin{center}

        \Large

        \vspace{-1.5em}

        \textbf{Remote}

        \vspace{.5em}
        \only<1>{\vspace{.5em}}

        \normalsize

        \begin{tikzpicture}
            \node[tiny-commit] at (0, 0)(a){};
            \node[tiny-commit] at (2, 0) (b){};
            \node[tiny-commit] at (4, 0) (c){};
            \node[tiny-commit, fill=TUDa-10a] at (6, 0) (d){};
            \node<3->[tiny-commit, fill=TUDa-8a] at (8, 0) (e){};
            \node<-2>[branch] at (10, 0) (master){origin/master};
            \node<3->[branch] at (12, 0) (master){origin/master};
            \draw[parent] (b) to (a);
            \draw[parent] (c) to (b);
            \draw[parent] (d) to (c);
            \draw<3->[parent] (e) to (d);
            \draw<-2>[ref-arc] (master) to (d);
            \draw<3->[ref-arc] (master) to (e);
        \end{tikzpicture}
        \rule{\textwidth}{.3mm}
        \Large

        \textbf{Local}

        \vspace{.5em}

        \normalsize
        \begin{tikzpicture}
            \node[tiny-commit] at (0, 0)(a){};
            \node[tiny-commit] at (2, 0) (b){};
            \node[tiny-commit] at (4, 0) (c){};
            \node<1>[tiny-commit, fill=TUDa-8a] at (6, 0) (d){};
            \node<2->[tiny-commit, fill=TUDa-10a] at (6, 0) (d){};
            \node<2->[tiny-commit, fill=TUDa-8a] at (8, 0) (e){};
            \node<1>[branch] at (10, 0) (master){master};
            \node<2->[branch] at (12, 0) (master){master};
            \draw[parent] (b) to (a);
            \draw[parent] (c) to (b);
            \draw[parent] (d) to (c);
            \draw<2->[parent] (e) to (d);
            \draw<1>[ref-arc] (master) to (d);
            \draw<2->[ref-arc] (master) to (e);
        \end{tikzpicture}

        \vspace{1em}

        \begin{onlyenv}<2>
            \bashcommandbob{git pull \textminus\textminus rebase}
        \end{onlyenv}
        \begin{onlyenv}<3>
            \bashcommandbob{git push}
        \end{onlyenv}
        \vspace{-1em}
    \end{center}
\end{frame}

\subsection{Rebase vs Merge}\label{subsec:rebase-vs-merge}
\begin{frame}[c]
    \centering
    \Large
    \textbf{Rebase vs Merge}
\end{frame}

\begin{frame}[c, fragile]
    \slidehead
    \begin{center}

        \Large

        \vspace{-1.5em}

        \textbf{Rebase}

        \vspace{.5em}
        \only<1>{\vspace{.5em}}

        \normalsize

        \begin{tikzpicture}
            \node[tiny-commit] at (0, 0)(a){};
            \node[tiny-commit] at (2, 0) (b){};
            \node[tiny-commit] at (4, 0) (c){};
            \node[tiny-commit, fill=TUDa-10a] at (6, 0) (d){};
            \node[tiny-commit, fill=TUDa-8a] at (8, 0) (e){};
            \node[branch] at (12, 0) (master){master};
            \draw[parent] (b) to (a);
            \draw[parent] (c) to (b);
            \draw[parent] (d) to (c);
            \draw[parent] (e) to (d);
            \draw[ref-arc] (master) to (e);
        \end{tikzpicture}
        \begin{onlyenv}<2->
            \rule{\textwidth}{.3mm}
            \Large

            \textbf{Merge}

            \vspace{.5em}

            \normalsize
            \begin{tikzpicture}
                \node[tiny-commit] at (0, 0)(a){};
                \node[tiny-commit] at (2, 0) (b){};
                \node[tiny-commit] at (4, 0) (c){};
                \node<2>[tiny-commit, fill=TUDa-8a] at (6, 0) (d){};
                \node<3->[tiny-commit, fill=TUDa-8a] at (6, -.75) (d){};
                \node<3->[tiny-commit, fill=TUDa-10a] at (6, .75) (e){};
                \node<4->[tiny-merge-commit] at (8, 0) (f){};
                \node<-3>[branch] at (10, 0) (master){master};
                \node<4->[branch] at (12, 0) (master){master};
                \draw[parent] (b) to (a);
                \draw[parent] (c) to (b);
                \draw[parent] (d) to (c);
                \draw<3->[parent] (e) to (c);
                \draw<4->[parent] (f) to (d);
                \draw<4->[parent] (f) to (e);
                \draw<-2>[ref-arc] (master) to (d);
                \draw<4->[ref-arc] (master) to (f);
            \end{tikzpicture}
        \end{onlyenv}
    \end{center}
\end{frame}

\begin{frame}[c]
    \slidehead
    \centering
    \Large
    \bashcommand{git config \textminus\textminus global pull.ff only}
\end{frame}

\subsection{Final Review}\label{subsec:final-review}
\begin{frame}[c]
    \slidehead
    \centering
    \Large
    \textbf{Final Review}
\end{frame}

\begin{frame}[c]
    \slidehead
    \large
    \textbf{Bewertungskriterien}
    \normalsize
    \begin{enumerate}
        \item<2-> Wie skaliert der Workflow? \only<3->{\hfill\textbf{nicht gut}}
        \item<4-> Können Bugs/Fehler einfach in den Master-Branch gelangen? \only<5->{\hfill\textbf{ja}}
        \item<6-> Kann man Fehler einfach rückgängig machen? \only<7->{\hfill\textbf{geht}}
        \item<8-> Erzeugt dieser Workflow eine neue unnötige, kognitive Überlastung für das Team? \only<9->{\hfill\textbf{nein}}
    \end{enumerate}
\end{frame}
